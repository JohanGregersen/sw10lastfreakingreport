For a system to support usage based insurance, two overall components are required. A client application supporting the collection and display of data, and a backend capable of storing all necessary information about users and how they drive. The implemented system however utilizes a second server in between the client and the storage server (See figure \ref{fig:system_model}). This server extracts all required information from the raw location data and passes it on to the storage server. For a final and scalable product, these computations should be done locally by the client. Using the intermediary server however allows us to simulate a finished product without mobile restrictions. Communication between components are done using JSON through REST APIs hosted on both servers. Data can be received or sent using the HTTP verbs GET and POST respectively.

\begin{figure}[tb]
\centering
\includegraphics[width=0.465\textwidth]{Pictures/system_model}
\caption{Composition of the system}
\label{fig:system_model}
\end{figure}