\subsubsection{Storage Server}\label{subsec:impl_storage_server}
Trip-processing is implemented on the API server, inside the logic layer. This opposes the design decision of optimizing the computational resources offered by the more powerful server. In the current system, it was considerably easier to implement this layer under the logic layer, and by monitoring the ongoing load on the API server, it seemingly manages it well for now. When the system has to serve more users, this has to be reworked to maximize the performance of the system. 

The trip-processing layer contain two extensive computational schemes. The first scheme computes all the measures and flags between every single GPS coordinate during the trip. This is the foundation for the entries inserted in the \texttt{TripFact} table, therefore this has to be accurate and trustworthy. This is where the design of six sub classes is valuable, because the appropriate objects can be chosen and sent to a \texttt{MeasureCalculator}, which contains mathematical formulas, like how to compute speed. Only the appropriate sub classes are sent as parameters, and the complete object is not thrown around between formulas.

The other scheme computes the required attributes in the \texttt{TripFact}, which is statistical groupings of the information stored in the \texttt{GPSFact} table. It uses these statistical attributes to analyze the drivers performance and compute the optimal tripscore, and the actual tripscore. The scoringmodel, used to compute the actual tripscore, can be seen in Section \ref{subsec:prereq}.