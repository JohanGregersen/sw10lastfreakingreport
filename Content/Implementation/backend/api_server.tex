\subsubsection{API Server}\label{subsec:impl_api_server}
As portrayed in Figure \ref{fig:backend_design} the API server consists of 3 layers described in this section.

The communication layer is fully implemented in C\# as a RESTful Web Service API, using the built-in .NET \texttt{ServiceModel} library. It is implemented following the design described in Section \ref{sec:api_server}. The communication layer includes thorough error-handling along with an error-reporting system. It is critical that the communication layer does not crash, because the access to the Drive-LaB backend will crash with it. The error-handling ensures that corrupted data being processed in the logic layer, does not cast exceptions back to the communication layer. Also, by using the .NET RESTful library, a series of error-handling tasks is conducted automatically. As example, if a client targets a non-existent service endpoint, or if a client use a wrong HTTP verb, the API will return a corresponding HTTP error code.

The logic layer is implemented in C\# using regular OOP-style programming. It contains many classes and methods to handle the variety of functionality handled by this layer. A majority of this functionality is to create appropriate C\# objects, either based on JSON data received from the communication layer, or data received in \texttt{DataRow} format from the DAL. DataRow is a .NET specific data type, designed to hold a row of data received from a database, which is what the DAL returns to the logic layer.  

The logic layer uses Json.NET, a popular JSON framework for .NET \citep{json_dot_net}. Using a third party framework to support JSON serialization and deserialization is necessary, because Mono does not contain Visual Studio libraries to support this task. Another library that is not implemented in Mono is \texttt{Device.Location}, which offers the type \texttt{GeoCoordinate} in C\#. It is used to store spatio-temporal data, and offers functionality like computing distance between two coordinates. A third party library called GeoCoordinatePortable offers this functionality while also being Mono-compatible \citep{geocoordinateportable}. Therefore, this library is used as substitute.

Last is the DAL. This layer is implemented using a combination of SQL and C\#. It makes use of \texttt{Npgsql} \citep{npgsql}, a .NET data provider for PostgreSQL, which makes it very easy to write SQL statements and C\# code in the same IDE. 