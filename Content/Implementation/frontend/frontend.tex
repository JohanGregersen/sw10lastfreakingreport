\subsection{Frontend}\label{subsec:frontend_implementation}
A goalpost for Drive-LaB is to be easily accessible. To fulfill this goal, Drive-LaB is developed for Android. As of 2015 Q2, over 80\% of shipped smartphones use the Android OS\ref{smartphone_market_share}. Choosing Android as platform has allowed for the release of Drive-LaB on Google Play, making it accessible for the vast majority of smartphone users. Furthermore, the implementation of Drive-LaB supports Android 4.0 – 6.0.1, making it compatible with over 97\% of current Android devices\ref{android_version_distribution}.
As mentioned in \ref{subsec:frontend_design}, the application has two responsibilities; data logging and data presentation. These responsibilities are mirrored in the structure of the application, seen in \textbf{SE FIGUR BLA}. The application consists of two major components. One is the graphical interface, consisting of 10 different Android activities\textbf{SOURCE}. The interface presents data for the user, and allows for interaction with \texttt{Location Service}, the second component.\texttt{Location Service} is an Android Service\textbf{SOURCE}. It runs separately from the rest of the application, in that it has its own process. It is however entirely controlled by the application (See \ref{subsubsec:service_communication}). The \texttt{Location Service} is responsible for all location logging. Running the service separately allows for a lifecycle independent of any Activities bound to it. In effect, the Android device can still be used normally. The application can be closed and wiped from memory, and the screen turned off, saving battery. None of these things will affect the \texttt{Location Service}.

\subsubsection{Service Communication}\label{subsubsec:service_communication}
The downside of having \texttt{Location Service} running in a process separate from the application, is that communication becomes non-trivial. With the chosen setup, there is no support for synchronous communication, method invocation, or even two-way communication. The alternative method of communication is illustrated in \textbf{FIGURE X}. Assuming the service is running, one establishes a connection using a \texttt{ServiceConnection}\textbf{SOURCE}. Upon successfully establishing a connection to the \texttt{Location Service}, one can use a \texttt{Messenger}\textbf{SOURCE} to send a message asynchronously. The \texttt{Location Service} can then use a \texttt{Handler}\textbf{SOURCE} to determine a course of action, depending on the message sent.

Often, the \texttt{Location Service} is required to respond to a message. Being unable to answer a message directly, \texttt{Location Service} utilizes \texttt{sendBroadcast}\textbf{SOURCE} which issues a message globally on the device. The receiving \texttt{Activity} can then listen for the message using a \texttt{BroadcastReceiver}\textbf{SOURCE} and act accordingly to the content of the message. 

\subsubsection{Location Logging}\label{subsubsec:location_logging}
\texttt{Location Service} is responsible for the continous retreival of location updates, when demanded. Retreiving location updates through Android is done using the Google Play services location APIs, specifically the \texttt{FusedLocationProviderApi}\textbf{SOURCE}. This API is able to automatically choose the best location provider, maximizing the possible precision and availability of location updates. Locations can therefore be based on both GPS, Cell-ID, and Wi-Fi. To achieve the desired quality and frequency of locations, these settings are used when requesting locations through the FusedLocationProviderApi:

\begin{itemize}
\item Desired interval: 1000ms
\item Fastest interval: 1000ms
\item Priority: High Accuracy
\end{itemize}

These settings enables Drive-Lab to receive locations exactly once every second whenever possible. Locations are furthermore pinpointed as exact as possible, regardless of battery consumption. This will usually result in GPS positions, as this is generally the more accurate option.