\subsection{Frontend}\label{subsec:frontend_implementation}
A goal for Drive-LaB is to be easily accessible. To fulfill this goal, Drive-LaB is developed for Android. As of 2015 Q2, over 80\% of shipped smartphones use the Android OS\citep{smartphone_market_share}. Choosing Android as platform has allowed for the release of Drive-LaB on Google Play\citep{google_play_drivelab}, making it accessible for the vast majority of smartphone users. Furthermore, the implementation of Drive-LaB supports Android 4.0 – 6.0.1, making it compatible with over 97\% of current Android devices\citep{android_version_distribution}.
As mentioned in Section \ref{subsec:frontend_design}, the application has two responsibilities; data collection and data presentation. These responsibilities are mirrored in the structure of the application, seen in Figure \ref{fig:frontend_components}. The application consists of two major components. One is the user interface, consisting of 10 different Android activities\citep{android_activity}. The interface presents data for the user, and allows for interaction with \texttt{Location Service}, the second component.\texttt{Location Service} is an Android Service\citep{android_service}. It runs separately from the rest of the application, in its own process. It is however entirely controlled by the user interface as seen in Section \ref{subsubsec:service_communication}. The \texttt{Location Service} is responsible for all location logging. Running the service separately allows for a lifecycle independent of any \texttt{Activity} bound to it. In effect, the Android device can still be used normally. The application can be closed, and the screen turned off, saving battery. This will not affect the \texttt{Location Service} in any way.

\begin{figure}[tb]
\centering
\includegraphics[width=0.465\textwidth]{Pictures/frontend_components}
\caption{The frontend consists of two components, running separately from each other}
\label{fig:frontend_components}
\end{figure}

\subsubsection{Service Communication}\label{subsubsec:service_communication}
The downside of having \texttt{Location Service} running in a process separate from the rest of the application, is that communication becomes non-trivial. With the chosen setup, there is no support for synchronous communication, method invocation, or even two-way communication. The alternative method of communication is illustrated in Figure \ref{fig:application_service_communication}. \texttt{Location Service} and \texttt{User Interface} represents a running instance of a \texttt{Service} and an arbitrary \texttt{Activity} respectively. For these components to communicate, an implementation of the interface \texttt{ServiceConnection} is required\citep{android_serviceconnection}.  The \texttt{ServiceConnection} is bound to the \texttt{Location Service} using \texttt{bindService}\citep{android_bindservice}. Upon a successful connection to the \texttt{Location Service}, the \texttt{ServiceConnection} instantiates a \texttt{Messenger}\citep{android_messenger} which is able to send messages asynchronously. Looking at the \texttt{Location Service} in figure \ref{fig:application_service_communication}, incoming messages are first caught by a \texttt{Handler}\citep{android_handler} class. By extending this class and overriding its \texttt{handleMessage} method, the \texttt{Location Service} is able to determine a course of action, depending on the message sent.

\begin{figure}[tb]
\centering
\includegraphics[width=0.465\textwidth]{Pictures/application_service_communication}
\caption{Two-way communication path between Location Service and User Interface}
\label{fig:application_service_communication}
\end{figure}

Often, the \texttt{Location Service} is required, not only to perform an action, but also respond to incoming messages. The \texttt{Location Service} is however unable to reference a calling \texttt{Activity}. Instead, the \texttt{Location Service} utilizes \texttt{sendBroadcast}\citep{android_sendbroadcast} which issues a message globally on the device. Returning to the \texttt{User Interface} side of figure \ref{fig:application_service_communication}, the receiving \texttt{Activity} can then listen for the message using a \texttt{BroadcastReceiver}\citep{android_broadcastreceiver} and act accordingly to the content of the broadcasted message. 

\subsubsection{Location Logging}\label{subsubsec:location_logging}
\texttt{Location Service} is responsible for the continuous retrieval of location updates. Retrieving location updates through Android is done using the Google Play services location APIs, specifically the \texttt{FusedLocationProviderApi}\citep{android_fusedlocationproviderapi}. This API is able to automatically choose the best location provider, maximizing the possible precision and availability of location updates. Locations can therefore be based on both GPS, Cell-ID, or Wi-Fi. To achieve the desired quality and frequency of locations, these settings are used when requesting locations through the FusedLocationProviderApi:

\begin{itemize}
\item Desired interval: 1000ms
\item Fastest interval: 1000ms
\item Priority: High Accuracy
\end{itemize}

These settings enables Drive-LaB to receive locations exactly once every second whenever possible. Locations are furthermore pinpointed as exact as possible, regardless of battery consumption. This will usually result in GPS positions, as this is generally the more accurate option.