\section{INTRODUCTION}\label{sec:intro}
Usage based car insurance is something that has been researched actively over the last decade. Several research papers describe ideas and models, trying to remedy concerns of fairness, privacy and ease of use. In spite of all the research, very few products have made it into the mainstream market. To the authors knowledge, no such product have made it to market in Denmark. Other countries are seeing experimental setups, but it seems no one is offering usage based insurance on a grand scale.
Usage based insurance or UBI is a complex product, having many potential problems and solutions, depending on the path taken. An example of a simple UBI product could be to measure distance driven by a user, and bill accordingly. Implementing such a simple system results in several difficult choices. Some of these could be:

\begin{itemize}
\item Whether to use custom measuring devices or rely on user equipment such as smartphones
\item Which technology to rely on for accurate positioning, and at which frequency
\item How to retrieve and store data logged by individual users
\item How to let users keep track of their insurance, understand and verify that they are being billed correctly
\end{itemize}

In this paper the authors implement a complete UBI system. We do this to answer problematiques associated with certain approaches to UBI. We try to answer the following questions:

\begin{itemize}
\item Is it possible to create a system that supports a fair and understandable metrification of driving styles?
\item Are modern smartphones adequate to support UBI?
\item How could a complete product look like?
\end{itemize}

\textbf{Fuld af postulater. Der skal findes kilder senere, og skrives rent.}