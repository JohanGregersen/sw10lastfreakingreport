\section{INTRODUCTION}\label{sec:intro}
Usage based car insurance is something that has been researched actively over the last decade. Insurance companies are showing interest in making UBI a reality, and some have even launched experimental products\citep{qbe_ubi}\citep{progressive_ubi}\citep{allstate_ubi}. There are however many problems on the road to achieving a fully functional UBI product. Several research papers describe ideas and models for final products, while others address specific concerns such as data quality\citep{art:insurtelematics} or privacy\citep{art:pripayd}. UBI products are however still sparse, and to the authors knowledge no one has offered UBI as a countrywide product, anywhere in the world.
For usage-based insurance to compete with traditional car insurance, there are many potential problems and solutions, depending on the chosen design. Considering a simple UBI product, insurance companies could measure distance driven by a user, and bill for mileage. Implementing such a simple UBI system still results in several non-trivial choices. Examples could be:

\begin{itemize}
\item Whether to use dedicated devices or rely on user equipment such as smartphones
\item Which technology to rely on for accurate positioning, and at which frequency
\item How to retrieve and store data logged by individual users
\item How to let users keep track of their insurance, understand and verify that they are being billed correctly
\end{itemize}

This paper presents the entire stack of a fully functional UBI system that anyone can use. We do this in an attempt to bring UBI one step further from ideas and models. Having a live UBI system allows for real-life experiments, and unique insight into problematiques associated with different approaches to UBI. In this paper the authors try to answer the following questions:

\begin{itemize}
\item Is it possible to create a UBI system that supports a fair and understandable metrification of driving styles?
\item Are modern smartphones adequate to support UBI?
\item How could a complete product look like?
\end{itemize}

The remainder of this paper describes the process leading towards answering these questions. In section \ref{sec:design} the system design is explained. Section \ref{sec:implementation} describes how the system has been implemented. Section \ref{sec:experiments} describes experiments made possible by releasing the system into a public domain. Finally, the authors provide answers to the problem statement based on the results of the experiments, followed by a conclusion in section \ref{sec:conclusion}.

\subsection{Prerequisites}\label{subsec:prereq}
Drive-LaB is based on the paper \textit{An Advanced Usage Based Insurance And Privacy-Secure Pricing Model}\citep{sw9_report}. The project features a metric-based scoringmodel for UBI, and focuses on being intuitive and understandable. Furthermore, it features an advanced data warehouse capable of storing all required data for supporting the described scoringmodel. Drive-LaB utilizes both the data warehouse and metric-based scoringmodel, although with certain improvements. The data warehouse implemented for Drive-LaB can be found in Appendices, Figure \ref{app:fig:newdatawarehouse}. Most notably, the \texttt{SubtTripFact} table has not been implemented in Drive-LaB. Instead, two new tables are introduced, namely \texttt{Competition Information} and \texttt{CompetingIn}.

The scoringmodel has not been altered, although its flexibility has allowed us to create a more fitting policy for the system. The policy used for experiments will be described in Section \ref{sec:experiments}. Each metric delinquency is divided into 8 different intervals, with different weights for each interval. Each metric is scored by the following algorithm:

$$
\left( \frac { \sum _{ i }^{ n }{ \left( { interval }_{ i }*\quad { weight }_{ i } \right)  }  }{ 100 }  \right) \quad -\quad 1
$$

This results in an aggregated weight. Roadtypes and Critical time period are scored linearly, calculated by multiplying the aggregated weight with the amount of delinquencies. Speeding, Accelerations, Brakes and Jerks are all evaluated polynomially. The aggregated weight is fed into a polynomial equation determined by the policy, resulting in a final aggregated weight which is multiplied with the amount of delinquencies.

\begin{align*}
ax^{y} + bx + c\quad \quad \quad \quad \quad \quad \quad \quad \quad \quad \quad \\
where\quad x = AggregatedWeight
\end{align*}

The full description of the scoringmodel is described in \citep{sw9_report} at Section 5.1.