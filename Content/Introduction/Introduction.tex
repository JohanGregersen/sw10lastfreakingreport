\section{INTRODUCTION}\label{sec:intro}
\begin{figure}[]
\centering
\includegraphics[width=0.48\textwidth]{Pictures/system_model2}
\caption{Composition of the system}
\label{fig:system_model}
\end{figure}

Usage-based car insurance (UBI) has been researched actively over the last decade. Insurance companies are showing interest in making UBI a reality, and some have even launched experimental products \citep{allstate_ubi} \citep{progressive_ubi} \citep{qbe_ubi}. Achieving a fully functional UBI product is a complex task involving difficult design choices and numerous technical challenges. Several research papers attempts to address specific concerns such as data quality \citep{art:insurtelematics} or privacy \citep{art:pripayd}. UBI products are still sparse, and to the authors knowledge no one has offered UBI as a countrywide product, anywhere in the world.
For usage-based insurance to compete with traditional car insurance, there are many potential problems and solutions, depending on the chosen design. Considering a simple UBI product, insurance companies could measure distance driven by a user, and bill for mileage accordingly. Implementing such a simple UBI system still results in several non-trivial choices. Examples could be:

\begin{itemize}
\item Whether to use dedicated devices or rely on user equipment such as smartphones
\item Which technology to rely on for accurate positioning, and at what frequency
\item How to retrieve and store data logged by individual users
\item How to let users keep track of their insurance, understand and verify that they are being billed correctly
\end{itemize}

This paper presents the entire stack of a fully functional UBI system that anyone can use. The authors attempt to move UBI away from ideas and models and one step closer to a market implementation. Having a live UBI system allows for real-life experiments, and offers unique insight into problematiques associated with different approaches to UBI. In this paper the authors try to answer the following questions:

\begin{itemize}
\item How could a complete product look like?
\item Is it possible to create a UBI system that supports a fair and understandable metrification of driving styles?
\item Are modern smartphones adequate to support UBI?

\end{itemize}

The remainder of this paper describes the process leading towards answering these questions. In Section \ref{sec:design} the system design is explained. Section \ref{sec:implementation} describes the implementation of the system. Section \ref{sec:experiments} describes experiments made possible by releasing the system into a public domain. Finally, the authors provide answers to the problem statement based on the results of the experiments, followed by a conclusion in Section \ref{sec:conclusion}.

\subsection{Prerequisites}\label{subsec:prereq}
Drive-LaB is based on \citep{sw9_report}. The project features a metric-based scoring model for UBI, and also focuses on being intuitive and understandable. Furthermore, it features an advanced data warehouse capable of holding all required data for supporting the described scoring model. Drive-LaB utilizes both the data warehouse and metric-based scoring model, although with certain improvements. The implemented data warehouse is shown in figure \ref{app:fig:newdatawarehouse}. What is most notable is that the subtrip fact table have not been implemented, however a Competition Information table and a CompetingIn table have been introduced.

The scoring model have not been altered, however the flexibility of the model have allowed us to create a more fitting policy for the system. The policy used for the experiments will be described in \ref{sec:experiments}.
