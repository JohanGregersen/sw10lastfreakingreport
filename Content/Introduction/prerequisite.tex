\subsection{Prerequisites}\label{subsec:prereq}
Drive-LaB is based on \citep{sw9_report}. The project features a metric-based scoring model for UBI, and also focuses on being intuitive and understandable. Furthermore, it features an advanced data warehouse capable of holding all required data for supporting the described scoring model. Drive-LaB utilizes both the data warehouse and metric-based scoring model, although with certain improvements. The implemented data warehouse is shown in Figure \ref{app:fig:newdatawarehouse}. What is most notable is that the subtrip fact table have not been implemented, however a \texttt{Competition Information} table and a \texttt{CompetingIn} table have been introduced.

The scoring model have not been altered, however the flexibility of the model have allowed us to create a more fitting policy for the system. The policy used for the experiments will be described in Section \ref{sec:experiments}. Each metric delinquency will be divided into 8 different intervals, with different weights for each interval. Each metric being scored by the following algorithm:

$$
\left( \frac { \sum _{ i }^{ n }{ \left( { interval }_{ i }*\quad { weight }_{ i } \right)  }  }{ 100 }  \right) \quad -\quad 1
$$

This results in a aggregated weight. Roadtypes and Critical time period are scored linearly, and simply calculated by multiplying the aggregated weight and the amount of delinquencies. Speeding, Accelerations, Brakes and Jerks are all evaluated polynomially however. The accumulated weight is run through a polynomial equation determined by the policy -ending in a final aggregated weight which is multiplied with the amount of delinquencies.

\begin{align*}
ax^{y} + bx + c\quad \quad \quad \quad \quad \quad \quad \quad \quad \quad \quad \\
where\quad x = AcummulatedWeight
\end{align*}

The full description of the tripscoring is described in \citep{sw9_report} at Section 5.1.