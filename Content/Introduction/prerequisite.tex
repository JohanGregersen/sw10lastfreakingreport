\subsection{Prerequisites}\label{subsec:prereq}
Drive-LaB is based on the paper \textit{An Advanced Usage Based Insurance And Privacy-Secure Pricing Model} \citep{sw9_report}. The project features a metric-based scoringmodel for UBI, and focuses on being intuitive and understandable. Furthermore, it features an advanced data warehouse capable of storing all required data for supporting the described scoringmodel. Drive-LaB utilizes both the data warehouse and metric-based scoringmodel, although with certain improvements. The data warehouse implemented for Drive-LaB can be found in Appendices, Figure \ref{app:fig:newdatawarehouse}. Most notably, the \texttt{SubtTripFact} table has not been implemented in Drive-LaB. Instead, two new tables are introduced, namely \texttt{Competition Information} and \texttt{CompetingIn}.

The scoringmodel has not been altered, although its flexibility has allowed us to create a more fitting policy for the system. The policy used for experiments will be described in Section \ref{sec:experiments}. Each metric delinquency is divided into 8 different intervals, with different weights for each interval. Each metric is scored by the following algorithm:

$$
\left( \frac { \sum _{ i }^{ n }{ \left( { interval }_{ i }*\quad { weight }_{ i } \right)  }  }{ 100 }  \right) \quad -\quad 1
$$

This results in an aggregated weight. Roadtypes and Critical time period are scored linearly, calculated by multiplying the aggregated weight with the amount of delinquencies. Speeding, Accelerations, Brakes and Jerks are all evaluated polynomially. The aggregated weight is fed into a polynomial equation determined by the policy, resulting in a final aggregated weight which is multiplied with the amount of delinquencies.

\begin{align*}
ax^{y} + bx + c\quad \quad \quad \quad \quad \quad \quad \quad \quad \quad \quad \\
where\quad x = AggregatedWeight
\end{align*}

The full description of the scoringmodel is described in \citep{sw9_report} at Section 5.1.