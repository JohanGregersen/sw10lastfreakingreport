\section{CONCLUSIONS}\label{sec:conclusion}
This paper has discussed the subject of usage-based insurance(UBI), and the metrics that are needed for a fair characterization of drivers. It describes the design and implementation of a data warehouse with focus on the customers' privacy while still enabling insurance companies to utilize a UBI billing scheme. Furthermore the authors suggests a highly customizable policy-dependent trip scoring model.
The goal of the paper was to create a system targeting insurance companies who wanted to enter a new and rather unexplored market. It provides not only a method for evaluating trips through different policies, but also provides the insurance company with the potential to collect a significant amount of statistical data on their customers.

Through experiments, we are able to demonstrate that even with spatially similar trips of near equal distance, our system is able to differentiate and price trips according to driving style. The paper offers an example policy in which drivers receive vastly different prices for their trips. These prices furthermore seems realistic, comparing them to the recorded driving patterns. The insurance company will however have to perform extensive testing themselves, to tune policies as they find best. 

The amount of metrics that have influence on the risk of driving cars are numerous, therefore some extensions of the system could be the addition of more metrics. Examples of potential metrics could include cornering, weather data, or the condition of the car at checkups. Other metrics that could be looked at would be available through an installed CAN bus in the car, and it would allow looking at fuel consumption, pollution, RPM, gear changes, etc.
A different direction for extensions could be improving the gamification of the system. The system in the paper is good because it makes it possible to compare on different segments, and at the same time it offers extra statistical data for the insurance company. By looking at comparisons between actual trip trajectories, this pose some interesting and complex issues.


\addtolength{\textheight}{-12cm}   % This command serves to balance the column lengths
                                  % on the last page of the document manually. It shortens
                                  % the textheight of the last page by a suitable amount.
                                  % This command does not take effect until the next page
                                  % so it should come on the page before the last. Make
                                  % sure that you do not shorten the textheight too much.