\section{CONCLUSIONS}\label{sec:conclusion}
The paper have discussed the entire system of Drive-LaB. It describes the design and implementation of the system described in \ref{sw9_report} with a variety of changes. The goal of the paper is to implement an experimental platform for usage-based insurance, where it is possible to test possible theories. The paper conducts a series of experiments to test whether or not the system is functional in an environment as close to commercial use as possible. Being functional in the environment means that the system can handle the traffic, the hardware used is optimal for the purpose. The findings of the experiments show that smartphones are irregular in logging GPS coordinates. 
Additional experiments have been made to test whether the scoring is fair and reliable. The users of the system scores lower as time progresses throughout the test. This is highly valuable from a insurance perspective -as it possibly lowers the risk assessment. One of the greatest feats of the system and this paper is that it offers information that have not been available for insurance companies before. It shows a driver profile in terms of metrics, and would possibly be able to characterise drivers with a higher risk of being involved in accidents.
The platform in it self provides a basis for numerous experiments involving GPS-coordinates and/or user interaction. It can be used simply as a GPS-logging system with smartphones as logging devices, and developer can add whatever data manipulation he wishes.
A possible extension for the existing system is improved the competition implementation in terms of diversity -possibly handling a wider variety kinds of competitions.


\addtolength{\textheight}{-12cm}   % This command serves to balance the column lengths
                                  % on the last page of the document manually. It shortens
                                  % the textheight of the last page by a suitable amount.
                                  % This command does not take effect until the next page
                                  % so it should come on the page before the last. Make
                                  % sure that you do not shorten the textheight too much.