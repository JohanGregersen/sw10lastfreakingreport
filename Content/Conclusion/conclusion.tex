\section{CONCLUSIONS}\label{sec:conclusion}
This paper has discussed the entire system of Drive-LaB. It describes the design and implementation of the system described in \textit{An Advanced Usage Based Insurance And Privacy-Secure Pricing Model}\citep{sw9_report} with a variety of changes and improvements. A goal of this paper is to propose a possible design and implementation for an experimental platform for usage-based insurance. This is clearly demonstrated in Section \ref{sec:design} and \ref{sec:implementation}, design and implementation of Drive-LaB, respectively.

The paper offers a system that can provide information that has not previously been available to insurance companies. It shows a driver profile in terms of comparable metrics, which can possibly be used to characterize and identify drivers with a higher risk of being involved in an accidents. This is proven by the experiments conducted in Section \ref{subsec:userprofiling}, Driver Profiling. Given a time period with this system in action, statistical data for risk assessment can be collected by monitoring which drivers actually are involved in an accident, and their driver profile could become a model to detect patterns in driving style that poses a higher risk. This type of data could benefit greatly from comprehensive collaboration with an insurance company, and is not part of this paper.

The paper presents a series of experiments on the implemented system. One experiment tests whether the system performs well in a real environment, comparable to commercial setting. Section \ref{subsec:userexp}, User Experiments, put the system to the test. In conclusion it performed sufficiently, with excess performance capabilities to spare in the live setting, with at least 10 drivers continuously using the system.

Another experiment was conducted to examine whether users of the system understands the metrification chosen in the scoringmodel described in Section \ref{subsec:prereq}, Prerequisites. Additionally, these users should return their experience, opinion and comments to using the system. Part of these responses should be whether they consider the scoring mechanism fair. The test is described in-depth in Section \ref{subsec:userexp}, User Experiments. Unfortunately, their response did not arrive in time, to make it into this paper. Consequentially, the authors are unable to answer whether the Drive-LaB supports a fair and understandable metrification of driver styles. 

On the other hand, it can be answered whether modern smartphones are adequate to support UBI. In Section \ref{subsec:expsystem}, System Experiments, two applicability tests are conducted to examine how five different smartphones and two high quality GPS devices perform in cooperation with Drive-LaB. The experiment is conclusive that smartphones are not adequate for usage-based insurance in collaboration with Drive-LaB. The test displays too diversified scores among different types of devices, but a list of possible solutions to the problem are stated as well. The experiment concludes that Drive-LaB should explore the possibility to implement model-based signal processing and outlier rejection schemes, to make the scoringmodel compute scores on more stringent trajectories. If this could be achieved, the scores might align themselves, or at least become accurate enough to consider a calibration mechanism.

The platform itself, provides a basis for numerous experiments involving GPS coordinates and/or user interaction. \textbf{fremtid for Drive-LaB?}

A possible extension for the existing system is to improved the competition implementation in terms of diversity -possibly handling a wider variety of competitions.


\addtolength{\textheight}{-12cm}   % This command serves to balance the column lengths
                                  % on the last page of the document manually. It shortens
                                  % the textheight of the last page by a suitable amount.
                                  % This command does not take effect until the next page
                                  % so it should come on the page before the last. Make
                                  % sure that you do not shorten the textheight too much.