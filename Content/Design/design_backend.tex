\subsection{Backend}\label{subsec:backend_design}
The backend of Drive-LaB is required to handle a large amount of incoming data. Whenever a user ends a trip, the logged data is packaged and sent to the backend for processing and storage. The backend is also required to respond to frontend requests for data used for visual presentation, as described in Section \ref{subsec:frontend_design}. These requirements are wrapped into a cloud-like architecture, where all data is stored on the Drive-LaB servers. Whenever a user attempts to access data on competitions or earlier trips, the application requests the data from the backend. This limits the amount of local storage required by the application and reduces resources needed for computation on the mobile device.

The backend of Drive-LaB is split across two servers. The less powerful of the two is exposed on the internet, hosting the communication layer, logical layer and data access layer. The other server is hosted only on the same local network as Server 1, with the trip-processing layer and physical storage layer, as seen by the standard UML component diagram in Figure \ref{fig:backend_design}. The two servers can be considered available resources, more than a profound design choice. The functionality design of these servers will now be explained, starting with Server 1, which will be called the API server. Server 2 will be called the storage server. No work has been put into security due to prioritization of resources and the commitment to complete the full range of functionality in the system.

\begin{figure}[tb]
\centering
\includegraphics[width=0.465\textwidth]{Pictures/backend_design}
\caption{Design of the Drive-LaB backend}
\label{fig:backend_design}
\end{figure}