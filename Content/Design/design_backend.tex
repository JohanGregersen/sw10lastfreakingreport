\subsection{Backend}\label{subsec:backend_design}
The backend of DriveLaB needs to handle a large amount of incoming data arriving in bulks. Whenever a user ends a trip, the entire trip will be packaged and sent to the backend for storage, logical computations, and storage. The backend also needs to respond to requests for data used in the visual presentation in the frontend-application, as described in section \ref{subsec:frontend_design}. These requirements are wrapped into a cloud-like architecture, where all data is stored on DriveLaB servers, and whenever the user want to examine their driving-performance, the application requests the data from the backend. This will highly limit the need for local storage in the application, reducing resources needed for both storage and computation on the smartphone. 

DriveLaB is built with two servers, one less powerful server exposed on the internet, hosting the communication layer for clients to communicate with, and one more powerful server hosted only on the same local network as server 1, handling the physical data storage and trip-processing  layer. The design for the functionality on these servers will now be explained, starting with server 1, which will be called the API-server. Server 2 will be called storage-server.