\subsection{Backend}\label{subsec:backend_design}
The backend of Drive-LaB needs to handle a large amount of incoming data. Whenever a user ends a trip, the entire trip will be packaged and sent to the backend for logical computations, and storage. The backend also needs to respond to requests for data used for visual presentation in the frontend-application, as described in section \ref{subsec:frontend_design}. These requirements are wrapped into a cloud-like architecture, where all data is stored on the Drive-LaB servers, and whenever the user want to examine their driving-performance, the application requests the data from the backend. This will highly limit the need for local storage in the application, reducing resources needed for both computation and storage on the smartphone. 

Drive-LaB is built with two servers, one less powerful server exposed on the internet, hosting the communication layer, logical layer and the data access layer, and one more powerful server hosted only on the same local network as server 1, with the trip-processing layer and physical storage layer, as seen by the standard UML component diagram in Figure \ref{fig:backend_design}. These two servers were the available resources, more than a profound design choice. The design for the functionality on these servers will now be explained, starting with server 1, which will be called the API server. Server 2 will be called the storage server. No work has been put into security due to prioritization of resources and the commitment to complete the full range of functionality in the system.

\begin{figure}[tb]
\centering
\includegraphics[width=0.465\textwidth]{Pictures/backend_design}
\caption{Design of the Drive-LaB backend}
\label{fig:backend_design}
\end{figure}