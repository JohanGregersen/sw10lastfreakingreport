A full-stack UBI system is by nature complex, as it involves a considerable amount of data from any driver using the system. This should however not be a concern for neither the end user or insurance company. Drive-LaB, as the UBI system is called, is therefore meant to be simple to understand and simple to use.
Drive-LaB is based on \textbf{SW9 PROJEKT REFERENCE}. The project features a metric-based scoring model for UBI, and also has a focus on being intuitive and understandable. Furthermore, it features an advanced data warehouse capable of holding all required data for supporting the described scoring model. Drive-LaB utilizes both the data warehouse and metric-based scoring model, although with certain improvements. These are described in \textbf{SEKTION REFERENCE HER}.
A goalpost this paper is to answer whether smartphones are suitable candidates for UBI. For this purpose, the frontend of Drive-LaB is designed as an mobile application. This also eliminates the need for a custom-built tracking device, increasing accessibility for anyone wanting to use Drive-LaB.
The design of Drive-LaB as a complete system can be seen in \ref{fig:system_model}. Three overall components exist. From the bottom up, we first have the storage server. In itself, it contains no logic, and simply holds the data warehouse. Above the storage server, an API server acts as interface for the frontend, and performs all required operations on incoming data, including database communication. The API server also serves data to the frontend on demand. At the top is the frontend application. It is responsible for trip tracking, and visual presentation of results calculated on the API server. As such the frontend allows end users to evaluate trips after driving them.

\begin{figure}[tb]
\centering
\includegraphics[width=0.465\textwidth]{Pictures/system_model}
\caption{Composition of the system}
\label{fig:system_model}
\end{figure}