\section{RELATED WORK}\label{sec:relatedwork}

P. Händel et. al. discusses the technology aspects of smartphone-based telematics, and highlights challenges in using smartphones as measurement probes\citep{art:insurtelematics}. They further suggest a number of metrics to differentiate between trips, and discuss the relevance and observability of these. The work is however only concerned with the smartphone and its possibilities. It does not concern itself with implementing a system to support usage based insurance.

P. Händel et. al. outlines a fully implemented system, capable of supporting usage based insurance\citep{art:smartphones_for_monitoring_and_ubi}. They further describe the release of said system and the collection of 250.000 kilometers worth of data over 10 months. The authors present findings, that data quality is a problem using smartphones for telemetry. They do however not follow up on whether the metric-based detection of driving style worked as intended. It is also unknown if the system had any effect on the driving style of end users.