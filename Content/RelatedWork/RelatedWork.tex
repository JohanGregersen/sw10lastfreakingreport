\section{RELATED WORK}\label{sec:relatedwork}

P. Händel et. al. discusses the technology aspects of smartphone-based telematics, and highlights challenges in using smartphones as measurement probes \citep{art:insurtelematics}. They further suggest a number of metrics to differentiate between trips, and discuss the relevance and observability of these. The work is only concerned with the smartphone and its possibilities. It does not concern itself with implementing a system to support usage-based insurance.

P. Händel et. al. outlines a fully implemented system, capable of supporting usage-based insurance \citep{art:smartphones_for_monitoring_and_ubi}. They further describe the release of said system and the collection of 250.000 kilometers worth of data in a span of 10 months. The authors present findings that prove data quality is a problem using smartphones for telematics. They do however not follow up on whether the metric-based detection of driving style worked as intended. It is also unknown if the system had any effect on the style of driving for the end users.