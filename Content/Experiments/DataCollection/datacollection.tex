\subsection{Data Collection}\label{sec:datacollection}
Releasing Drive-LaB publically has allowed for the collection of a sizeable dataset. For this report, 51 people participated, contributing a total of 581 trips spanning 13.075 kilometers. The trips are all logged between May 1st, 2016 and June 11th, 2016. All installations of Drive-LaB uses the same setup for logging locations, ensuring a uniform premise across all contributed trips. As mentioned in Section \ref{subsubsec:location_logging}, the requested sample rate is set at 1Hz, but conditions such as hardware limitations and poor signal can block this from being possible. As such, the collected data varies in sample rate. The total average of the dataset is 0.63Hz, still making it high-frequency, but lower than the desired 1Hz.

\subsubsection{Data}\label{subsec:data}
Trips logged through Android devices contain a set of latitudes, longitudes and timestamps with 1Hz granularity. The Drive-LaB application does not store data permanently and therefore only holds this data in memory. As a trip is sent to the storage server, all data is received by the API server, described in Section \ref{sec:api_server}. When the data reach the physical storage layer on the storage server, all data is stored in tables corresponding to the data warehouse schema described in Figure \ref{app:fig:newdatawarehouse} in Appendices, which is based on \citep{sw9_report}. 

\subsubsection{Policy}\label{subsec:policy}
The trips have been evaluated based on a policy throughout the test. The policy is described in Tables \ref{tab:roadtypevalues}, \ref{tab:crittimevalues}, \ref{tab:speedingvalues}, \ref{tab:accelerationvalues} and \ref{tab:jerkbrakenvalues}. The base values for the delinquencies are described in Table \ref{tab:basevalues}. The most notable change is the polynomial scoring have been omitted, as it is uncertain how the equation directly affects the scoring of the system. It should be revised when more information about the effect is present.

\begin{table}
    \centering
    \begin{tabular}{|ll|}
    \hline
    \rowcolor{tablegreen}
    \textbf{Roadtype} & \textbf{Weight} \\ \hline
    Motorway          & 0.8               \\
    Trunk             & 0.9               \\
    Primary           & 0.95               \\
    Secondary         & 1.05            \\
    Tertiary          & 1.1             \\
    Unclassified      & 1.1             \\
    Residential       & 1.2             \\
    Service           & 1.2             \\ \hline
    \end{tabular}
    \caption{Roadtypes with weights}
    \label{tab:roadtypevalues}
\end{table}

\begin{table}
    \centering
    \begin{tabular}{|llll|}
    \hline
    \rowcolor{tablegreen}
    \textbf{Active days} & \textbf{Start} & \textbf{End} & \textbf{Weight} \\ \hline
    Monday - Friday      & 07:00:00       & 09:00:00     & 1.16             \\
    Monday - Friday      & 15:00:00       & 17:00:00     & 1.12            \\
    Saturday - Sunday    & 09:00:00       & 13:00:00     & 1.02           \\
    Saturday - Sunday    & 20:00:00       & 23:59:59     & 1.12            \\
    Saturday - Sunday    & 00:00:00       & 00:04:00     & 1.325             \\\hline
    \end{tabular}
    \caption{Critical time intervals with weights}
    \label{tab:crittimevalues}
\end{table}

\begin{table}
    \centering
    \begin{tabular}{|ll|}
    \hline
    \rowcolor{tablegreen}
    \textbf{Interval (\%)}   & \textbf{Weight} \\ \hline
    {[}0, 10{[}        & 1.3                   \\
    {[}10, 20{[}       & 1.4                   \\
    {[}20, 30{[}       & 1.5                   \\
    {[}30, 40{[}       & 1.6                   \\
    {[}40, 50{[}       & 1.7                   \\
    {[}50, 60{[}       & 1.8                   \\
    {[}60, 70{[}       & 1.9                   \\
    {[}70, $\infty${]} & 2.0                   \\ \hline
    \end{tabular}
    \caption{Speeding intervals with weights}
    \label{tab:speedingvalues}
\end{table}

\begin{table}
    \centering
    \begin{tabular}{|ll|}
    \hline
    \rowcolor{tablegreen}
    \textbf{Interval (m/s)} & \textbf{Weight} \\ \hline
    {[}6, 7{[}              & 1.05             \\
    {[}7, 8{[}              & 1.10             \\
    {[}8, 9{[}              & 1.175             \\
    {[}9, 10{[}             & 1.275             \\
    {[}10, 11{[}            & 1.40            \\
    {[}11, 12{[}            & 1.55             \\
    {[}12, 13{[}            & 1.725             \\
    {[}13, $\infty${]}      & 2.0             \\ \hline
    \end{tabular}
    \caption{Acceleration with weights}
    \label{tab:accelerationvalues}
\end{table}

\begin{table}
    \centering
    \begin{tabular}{|ll|}
    \hline
    \rowcolor{tablegreen}
    \textbf{Interval (m/s)} & \textbf{Brake Weight} \\ \hline
    {[}8, 9{[}              & 1.05 			             \\
    {[}9, 10{[}             & 1.10			              \\
    {[}10, 11{[}            & 1.175   		           \\
    {[}11, 12{[}            & 1.275  		            \\
    {[}12, 13{[}            & 1.40   		          \\
    {[}13, 14{[}            & 1.55    		          \\
    {[}14, 15{[}            & 1.725  		            \\
    {[}15, $\infty${]}      & 2.0    	   	       \\ \hline
    \end{tabular}
    \caption{Jerks and brakes with weights}
    \label{tab:jerkbrakenvalues}
\end{table}

\begin{table}
    \centering
    \begin{tabular}{|ll|}
    \hline
    \rowcolor{tablegreen}
    \textbf{Action} & \textbf{Base weight} \\ \hline
    Acceleration    & 40                   \\
    Brake           & 65                   \\
    Jerk            & 15                 \\ \hline
    \end{tabular}
    \caption{Base weights for accelerations, brakes and jerks}
    \label{tab:basevalues}
\end{table}

%
%\begin{table}
%    \centering
%    \begin{tabular}{ll}
%    \textbf{Parameter} & \textbf{Weight} \\ \hline
%    A                  & 1            \\
%    B                  & 1            \\
%    C                  & 0               \\
%    Polynomial degree  & 1            \\ \hline
%    \end{tabular}
%    \caption{Weights for all polynomial functions}
%    \label{tab:polyvalues}
%\end{table}