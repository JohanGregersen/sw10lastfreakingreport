\begin{abstract}
Usage-based insurance (UBI) is currently surfacing both in research and within insurance companies. There are a lack of actual described UBI products, and those that exist are experimental and limited to small customer segments. Insurance companies are showing a clear interest in entering the market, but UBI as a product is complex, and little research exists when it comes to completely implemented products. In this paper, the authors describe the design, implementation and experimentation on Drive-LaB, a fully functional UBI platform. Drive-LaB lets users collect spatio-temporal data with their smartphone. The system uses this data to identify driving style and environmental context, to allow risk assessment associated with car insurance. Drive-LaB is supported by a complex backend system featuring an advanced data warehouse and computational logic to identify driver styles. It also offers an easy-to-use Android application frontend, allowing users to log trips and see detailed statistics on completed trips. Drive-LaB has been used for experiments, collecting more than 13.000 kilometers worth of data in roughly one month. This data has been used to validate the platform and display how the system performs in a realistic setting.
\end{abstract}