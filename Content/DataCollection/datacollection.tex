\section{DATA COLLECTION}\label{sec:datacollection}
Releasing Drive-LaB publically has allowed for the collection of a sizeable dataset. For this report, \textbf{NUMBER} people participated, contributing a total of \textbf{NUMBER} trips spanning \textbf{NUMBER} kilometers. The trips are all logged between May 1st, 2016 and May 31st, 2016. All installations of Drive-LaB uses the same setup for logging locations, meaning that the setup is uniform across all contributed trips (See \ref{subsubsec:location_logging}). As mentioned, the requested sample rate is set at 1hz, but conditions such as hardware limitations and poor signal can block this from being possible. As such, the collected data varies in sample rate. The total average of the dataset is \textbf{NUMBER}, still making it high-frequency, but lower than the desired 1hz.

\subsection{Data}\label{subsec:data}
Trips logged through Android devices contain a set of latitudes, longitudes and timestamps with second-precision. Drive-LaB does not store data permanently and therefore only holds this data in memory, namely using the Android Location class (See \ref{android_location}). As trips are sent to the storage server, all data is treated by the intermediate server, described in \ref{subsec:api}. Upon reaching the storage server, all data matches the data warehouse described in \ref{sw9_report}. \textbf{Tilføj mere - Der er ingen quality, vi bruger competingin, vi stjæler IMEI, og tilføjer localtripid} 